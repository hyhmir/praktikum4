\documentclass[12pt]{article}
\usepackage[slovene]{babel}
\usepackage[utf8]{inputenc}
\usepackage[T2A]{fontenc}
\usepackage{amsmath}
\usepackage{amsfonts}
\usepackage{amssymb}
\usepackage[version=4]{mhchem}
\usepackage{stmaryrd}
\usepackage{graphicx}
\usepackage[export]{adjustbox}
\graphicspath{ {./images/} }
\usepackage{physics}
\usepackage{geometry}
\geometry{left=2cm,right=2cm,top=2cm,bottom=2cm}

\title{\textbf{Uklon Svetlobe}}
\author{Samo Krejan}
\date{april 2025}

\begin{document}
\maketitle

\section{Uvod}

Svetloba se na robu ukloni. Če jo pošljemo skozi $N$ rež z debelino $D$ in na razmiku $d$ dobimo odvisnost svetlobnega toka od kota kota \ref{enačba1}

\begin{equation}
    I(\theta) = I_0 \left( \frac{\sin(\pi D \sin(\theta)/\lambda)}{\pi D \sin(\theta)/\lambda}\frac{\sin(N\pi d \sin(\theta)/\lambda)}{\sin(\pi d \sin(\theta)/\lambda)} \right)^2
    \label{enačba1}
\end{equation}
kjer je $\lambda$ valovna dolžina svetlobe. Pri majhnih kotih lahko aproksimiramo $\sin(\theta) = \theta$, kot pa kot $\theta = x/s$, kjer je $x$ oddaljenost od središčne lege, $s$ pa razdalja od reže do zaslona. Na okrogli odprtini dobimo kolobarjast vzorec (Fresnelove cone). V temu primeru velja, da se minimum ali maksimum pojavi pri pogoju \ref{feri}:

\begin{equation}
    \frac{2\pi R_n^2}{4\lambda\zeta}=\frac{n\pi}{2}
    \label{feri}
\end{equation}
kjer so $R_n$ polmeri fernelovih con.


\section{Potrebščine}
\begin{itemize}
    \item HeNe laser z valovno dolžino 633 nm, nosilna plošča za laser in translator za zaslone,
    \item par prizem v nosilcu za razširitev žarka,
    \item zasloni z odprtinami, leča z nosilcem, ravno ogledalo z nosilcem,
    \item x translator z montiranim fotodetektorjem in pretvornikom signalov,
    \item prenosnik z ustrezno programsko opremo.
\end{itemize}


\section{Naloga}

\begin{itemize}
    \item Izmeri uklonsko sliko svetlobe za zasloni z režami. Uporabi zaslone z 1, 2, 3, 5 in 10 režami. Določi relativneintenzitete uklonskih slik. Določi širino rež $D$ in razdalje med njimi $d$.
    \item Opazuj uklon na okrogli odprtini. Določi premer odprtine $2R$.
\end{itemize}
\newpage
\section{Rezultati in analiza}

Izmerili smo odvisnost moči svetlobe od premika $x$ pri uklonu na 1, 2, 3, 5, 10 režah. Meritve smo prikazali na grafih %\ref{1}, \ref{2}, \ref{3}, \ref{5}, \ref{10}
Podatke smo 'pofittali' s teoretičnim modelom in dobljene parametre napisali v tabelo %tabela ref


\end{document}