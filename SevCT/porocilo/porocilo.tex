\documentclass[10pt]{article}
\usepackage[slovene]{babel}
\usepackage[utf8]{inputenc}
\usepackage[T2A]{fontenc}
\usepackage{amsmath}
\usepackage{amsfonts}
\usepackage{amssymb}
\usepackage[version=4]{mhchem}
\usepackage{stmaryrd}
\usepackage{graphicx}
\usepackage[export]{adjustbox}
\graphicspath{ {./images/} }
\usepackage{physics}
\usepackage{geometry}
\geometry{left=2cm,right=2cm,top=2cm,bottom=2cm}



\title{Sevanje črnega telesa}
\author{Samo Krejan}
\date{marec 2025}


\begin{document}
\maketitle

\section{Uvod}

Gostoto energije elektromagnetnega valovanja z neko frekvenco $\nu$ v votlini pri temperaturi $T$ določa Planckova formula \ref{planck}.

\begin{equation}
    w(\nu, T) = \frac{8\pi h \nu^3}{c^3} \frac{1}{\exp(h\nu/kT) -1},
    \label{planck}
\end{equation}
kjer je $h$ Planckova konstanta in $c$ hitrost svetlobe v vakumu. Če v tako votlino naredimo majhno luknjico je to kar najboljši približek sevanja črnega telesa. Gostota energijskega toka skozi tako luknjico je \ref{gostota}:

\begin{equation}
    j(\nu, T) = \frac{1}{4} c w(\nu,T)
    \label{gostota}
\end{equation}
Svetloba ki seva iz luknjice sledi Lambertovem kosinusnem zakonu (intenziteta je sorazmerna s kosinusom kota pod katerim opazujemo). Tudi volframska nitka v žarnici je dober približek črnega telesa, kot je tudi sonce, ki seva kot črno telo pri temperaturi $5800\ K$. S to vajo bomo merili sevanje volframske nitke v halogenski žarnici, ki ji lahko spreminjamo temperaturo v zelo širokem obsegu. Z absolutnim merilnikom sevanja bomo določili celoten energijski tok, ki ga seva žarnica in ga nato primerjali z močjo, ki jo troši.


\section{Naloga}

\begin{itemize}
    \item Izmerite odvisnost svetlobnega toka halogenske žarnice v razponu od rahlega žarjenja do maksimalne moči. Pri tem merite tudi moč, ki se troši na žarnici,
    \item narišite graf celotne izsevane moči kot funkcijo izsevane moči,
    \item določite električno upornost žarnice kot funkcijo temperature,
    \item narišite graf razmerja med - skozi $Si$ okno - prepuščenim in nemotenim svetlobnim tokom kot funkcijo temperature žarilne nitke.
\end{itemize}


\section{Potrebščine}

\begin{itemize}
    \item Merilec električne moči (wattmeter) in električni multimeter,
    \item halogenska žarnica nazivne moči $30W$ z nazivno barvno temperaturo $2700K$,
    \item nastavljivi transformator - variac,
    \item merilnik sevanja,
    \item plošča iz kristalnega silicija.
\end{itemize}

\section{Rezultati in analiza}



\end{document}