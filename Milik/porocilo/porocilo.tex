\documentclass[12pt]{article}
\usepackage[slovene]{babel}
\usepackage[utf8]{inputenc}
\usepackage[T2A]{fontenc}
\usepackage{amsmath}
\usepackage{amsfonts}
\usepackage{amssymb}
\usepackage[version=4]{mhchem}
\usepackage{stmaryrd}
\usepackage{graphicx}
\usepackage[export]{adjustbox}
\graphicspath{ {./images/} }
\usepackage{physics}
\usepackage{geometry}
\geometry{left=2cm,right=2cm,top=2cm,bottom=2cm}

\title{\textbf{Millikanov poskus}}
\author{Samo Krejan}
\date{maj 2025}

\begin{document}
\maketitle

\section{Uvod}

Millikanov poskus je zgodovinsko zelo pomemben, saj je prvi določil vrednost osnovnega naboja $e_0$. To je dosegel tako, da je obravnaval nabite oljne kapljice v zraku pod uplivom električnega polja $E$. Ko kaplica neha pospeševati, nanjo delujejo tri sile, katerih vsota je enaka $0$. Te sile so; gravitacijska sila, sila upora (Stokesova sila)

\section{Potrebščine}


\section{Naloga}


\section{Rezultati in analiza}


\end{document}