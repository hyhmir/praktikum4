\documentclass[12pt]{article}
\usepackage[slovene]{babel}
\usepackage[utf8]{inputenc}
\usepackage[T2A]{fontenc}
\usepackage{amsmath}
\usepackage{amsfonts}
\usepackage{amssymb}
\usepackage[version=4]{mhchem}
\usepackage{stmaryrd}
\usepackage{graphicx}
\usepackage[export]{adjustbox}
\graphicspath{ {./images/} }
\usepackage{physics}
\usepackage{geometry}
\geometry{left=2cm,right=2cm,top=2cm,bottom=2cm}

\title{\textbf{Millikanov poskus}}
\author{Samo Krejan}
\date{maj 2025}

\begin{document}
\maketitle

\section{Uvod}

Millikanov poskus je zgodovinsko zelo pomemben, saj je prvi določil vrednost osnovnega naboja $e_0$. To je dosegel tako, da je obravnaval nabite oljne kapljice v zraku pod uplivom električnega polja $E$. Ko kaplica neha pospeševati, nanjo delujejo tri sile, katerih vsota je enaka $0$. Te sile so; gravitacijska sila, sila upora (Stokesova sila) in električna sila. Električno polje lahko kaže v smeri gravitacijskega pospeška (+) ali pa proti njemu (-). Ravnovesje sil se izrazi kot \ref{sile}:

\begin{equation}
    \frac{4\pi r^3}{3} (\rho_0-\rho_z)g \pm n e_0 E = 6\pi r \eta v_\pm
    \label{sile}
\end{equation}
Tu je $\rho_0$ gostota olja, $\rho_z$ gostota zraka, $E = U/d$ jakost električnega polja, $e_0$ osnovni naboj, $n$ število osnovnih nabojev v kapljici in $\eta$ viskoznost zraka. Če za posamezno kaplico izmerimo hitrost v polju, usmerjenem dol in gor, lahko določimo radij kapljice \ref{radi}, ter naboj kapljice \ref{naboj} kot:

\begin{equation}
    r = \sqrt{\frac{9 \eta (v_+ + v_-)}{4g(\rho_0-\rho_z)}}
    \label{radi}
\end{equation}

\begin{equation}
    ne_0 = \frac{3\pi r \eta}{E}(v_+ - v_-)
    \label{naboj}
\end{equation}

\section{Potrebščine}

\begin{itemize}
    \item Millikanov aparat: kondenzator, razpršilec z oljem, LED za osvetljevanje,
    \item mikroskop s kamero, ki je priključena na računalnik,
    \item usmernik za 300V,
    \item preklopnik smeri napetosti,
    \item voltmeter.
\end{itemize}


\section{Naloga}

\begin{enumerate}
    \item Izmeri hitrosti gibanja kapljiv v električnem in gravitacijskem polju,
    \item iz meritve izračunaj hitrost kapljic in njihov naboj, ter določi osnovni naboj.
\end{enumerate}


\section{Rezultati in analiza}




\end{document}