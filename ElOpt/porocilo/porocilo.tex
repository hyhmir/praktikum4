\documentclass[12pt]{article}
\usepackage[slovene]{babel}
\usepackage[utf8]{inputenc}
\usepackage[T2A]{fontenc}
\usepackage{amsmath}
\usepackage{amsfonts}
\usepackage{amssymb}
\usepackage[version=4]{mhchem}
\usepackage{stmaryrd}
\usepackage{graphicx}
\usepackage[export]{adjustbox}
\graphicspath{ {./images/} }
\usepackage{physics}
\usepackage{geometry}
\geometry{left=2cm,right=2cm,top=2cm,bottom=2cm}

\title{\textbf{Elektrooptični pojav}}
\author{Samo Krejan}
\date{marec 2025}

\begin{document}
\maketitle

\section{Uvod}

Z zunanjim električnim poljem lahko navadno vplivamo na strukturo snovi. V kristalih naprimer tako lahko spremenimo obliko osnovne celice, v tekočinah lahko pride do spremembe gostote in orientacijskega orientiranja molekul.

\section{Potrebščine}


\section{Naloga}


\section{Rezultati in analiza}


\end{document}